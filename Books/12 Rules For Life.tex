\documentclass{article}
\usepackage{multicol}
\usepackage{ragged2e}
\title{12 Rules For Life}
\author{Grayson Wills}
\begin{document}
\maketitle

\section{Stand up straight with your shoulders back}
    \begin{multicols}{2}
        \begin{center}
            \textbf{Overview}
        \end{center}
        
        The text goes pretty deep into how different species assert and show dominance over each other, and what that particularly means for the animal kingdom, and it's pretty interesting stuff. He talks about how lobsters control their territory and perform rituals to show they are the alpha male. The main premise it to compare to how human beings are also animals and need to use similar methods within their own range to show control. We don't have claws or fangs, but we are unique in the sense that we are bipedal, and so standing tall with our shoulders back is our way of showing command over the other animals in the animal kingdom

        \begin{center}
            \textbf{Personal Thoughts}

        \end{center}
        

            I get what he's saying, and I do find it important to do this. I need to fix up some posture issues that I currently have, since there is still a bit of that muscle imbalance, but I am on my way to achieving this goal. This is a good rule that I plan to utilize for myself.
        
    \end{multicols}
 
\section{Treat yourself like someone you are responsible for helping}
       \begin{multicols}{2}
        \begin{center}
            \textbf{Overview}
        \end{center}
        
        It's interesting how on the nose he is about this. He talks about how human beings have a natural tendency to take care of OTHER people, including their animals, but won't help themselves. He uses medicine as an example. He says that if a person owns a dog, they are more likely to give a dog life saving medicine than they are to give themselves life saving medicine. What an interesting and thought-provoking idea. 

        \begin{center}
            \textbf{Personal Thoughts}
        \end{center}
        
        Like I said, right on the nose. There is nothing in this chapter that doesn't make sense to me. It's quite clear that how I used to be about other people, i.e. have that stupid hero complex, is nothing short of destructive. That doesn't mean that I can't help other people when they need it, but it does mean that I also have to take care of myself and put my needs first from time to time, or even most of the time. Sacrifice is exactly that... sacrifice, and if you keep giving parts of yourself, eventually you won't have anything left over to receive.
        
    \end{multicols}
    
\section{Make friends with people who want the best for you}
     \begin{multicols}{2}
        \begin{center}
            \textbf{Overview}
        \end{center}
        
        There are people in life who are going to get incredibly jealous and try to pull you back down to their level. There are also people who are going to try and pull you back because you're going down the wrong path. Determining the difference between those people is and incredibly tricky line, and it's one of the finest lines known to man. How do you tell the difference between care and jealousy when you're making progress in something? Progress is progress, whether the activity is productive or not... there is a reason people get really good at partying and drinking, and it's because they make consistent progress towards it. It's strange to think about.

        \begin{center}
            \textbf{Personal Thoughts}
        \end{center}
        
        Luckily for me, I don't seem to have trouble finding these types of people. They seem to be abundantly everywhere, and I think it's because I am one of those people that want the best for others and don't get jealous or angry (anymore) when they succeed. It's a liberating thing to see how when you do something, it can be reflected back at you for one reason or another.
        
    \end{multicols}

\section{Compare yourself to who you were yesterday, not to who someone else is today}
    \begin{multicols}{2}
        \begin{center}
            \textbf{Overview}
        \end{center}
        
        You are in charge of yourself. In the wise words of Bruce Lee, I am not here to live up to your expectations and you are not here to live up to mine. It's quite interesting how once you let yourself go and start focusing on your own progress and not the progress of others, you tend to start seeing linear and even exponential progress. 

        \begin{center}
            \textbf{Personal Thoughts}
        \end{center}
        
        This it true to many degrees. Watching TV, scrolling through Instagram... you don't realize how often you compare yourself to others until you start to weed yourself off of it, and you realize that everything that you've been doing has been filled with jealousy, hopelessness, and a sense of loss. It's quite interesting how in just a couple months of getting off of those systems, you go ahead and turn it all around in one go, and you can definitely keep it going for a while it seems like. I think that's an interesting concept, and it's not something that I've ever thought about until I was one of those drones. But now that I know, I can take measures to prevent it from ever happening again. And for the record... I'm loving it... Oh I'm in love with the progress. I didn't realize how much I missed the momentum, the wildly racing thoughts and ideas, and the ability to think this critically or make decisions on the fly. This is going to be something that I'll enjoy for a while, and I hope to keep loving it.
        
    \end{multicols}

\section{Do not let your children do anything that makes you dislike them}
        \begin{multicols}{2}
        \begin{center}
            \textbf{Overview}
        \end{center}
        
        It is you job to raise children into being people that you desire them to be, hopefully ones that are better than you and are able to sidestep the mistakes that you've made along the way. It's a tough world, and your children need to be able to handle it. You need to fill them with love, discipline, and the courage to think. It's your job to make them independent and able to achieve greatness on their own, rather than be dependent on you and subject to their every impulse and vice. It's a hard line as well when you need to punish... How do you decide the severity of the punishment? Is it physical? Is it just a stern look? The simple rule: The punishment should fit the crime.

        \begin{center}
            \textbf{Personal Thoughts}
        \end{center}
        
        I can't speak to this. I am not a father... as of writing this, I am 23 years old with no dating prospects in sight. I'm just hoping to be a good father one day, so that means right now I have to work to be a person that my children can look up to.
    \end{multicols}

\section{Set your house in perfect order before you criticize the world}
        \begin{multicols}{2}
        \begin{center}
            \textbf{Overview}
        \end{center}
        
        This has multiple points to it, but the main one is organization and discipline in the home. Before you can start to make the world a better place, you have to woork to set your own life in order, and that starts in the home.

        \begin{center}
            \textbf{Personal Thoughts}
        \end{center}
        
        I definitely struggle with this. The past two years I have done a poor job of keeping myself organized. I need to do better!
    \end{multicols}
    
\section{Pursue what is meaningful (not what is expedient)}
        \begin{multicols}{2}
        \begin{center}
            \textbf{Overview}
        \end{center}
        
        Filler words 

        \begin{center}
            \textbf{Personal Thoughts}
        \end{center}
        
        filler words
    \end{multicols}
    
\section{Tell the truth – or, at least, don't lie}
        \begin{multicols}{2}
        \begin{center}
            \textbf{Overview}
        \end{center}
        
        Filler words 

        \begin{center}
            \textbf{Personal Thoughts}
        \end{center}
        
        filler words
    \end{multicols}
    
\section{Assume that the person you are listening to might know something you don't}
        \begin{multicols}{2}
        \begin{center}
            \textbf{Overview}
        \end{center}
        
        Filler words 

        \begin{center}
            \textbf{Personal Thoughts}
        \end{center}
        
        filler words
    \end{multicols}

\section{Be precise in your speech}
        \begin{multicols}{2}
        \begin{center}
            \textbf{Overview}
        \end{center}
        
        Filler words 

        \begin{center}
            \textbf{Personal Thoughts}
        \end{center}
        
        filler words
    \end{multicols}
    
\section{Do not bother children when they are skateboarding}
        \begin{multicols}{2}
        \begin{center}
            \textbf{Overview}
        \end{center}
        
        Filler words 

        \begin{center}
            \textbf{Personal Thoughts}
        \end{center}
        
        filler words
    \end{multicols}
    
\section{Pet a cat when you encounter one on the street}
        \begin{multicols}{2}
        \begin{center}
        \textbf{Overview}
        \end{center}
        
        Filler words 

        \begin{center}
            \textbf{Personal Thoughts}
        \end{center}
        
        filler words
    \end{multicols}
\section{Personal Analysis}
\end{document}
