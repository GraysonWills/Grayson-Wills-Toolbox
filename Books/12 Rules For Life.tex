\documentclass{article}
\usepackage{multicol}
\usepackage{ragged2e}
\title{12 Rules For Life}
\author{Grayson Wills}
\begin{document}
\maketitle

\section{Stand up straight with your shoulders back}
    \begin{multicols}{2}
        \begin{center}
            \textbf{Overview}
        \end{center}
        
        The text goes pretty deep into how different species assert and show dominance over each other, and what that particularly means for the animal kingdom, and it's pretty interesting stuff. He talks about how lobsters control their territory and perform rituals to show they are the alpha male. The main premise it to compare to how human beings are also animals and need to use similar methods within their own range to show control. We don't have claws or fangs, but we are unique in the sense that we are bipedal, and so standing tall with our shoulders back is our way of showing command over the other animals in the animal kingdom

        \begin{center}
            \textbf{Personal Thoughts}

        \end{center}
        

            I get what he's saying, and I do find it important to do this. I need to fix up some posture issues that I currently have, since there is still a bit of that muscle imbalance, but I am on my way to achieving this goal. This is a good rule that I plan to utilize for myself.
        
    \end{multicols}
 
\section{Treat yourself like someone you are responsible for helping}
       \begin{multicols}{2}
        \begin{center}
            \textbf{Overview}
        \end{center}
        
        It's interesting how on the nose he is about this. He talks about how human beings have a natural tendency to take care of OTHER people, including their animals, but won't help themselves. He uses medicine as an example. He says that if a person owns a dog, they are more likely to give a dog life saving medicine than they are to give themselves life saving medicine. What an interesting and thought-provoking idea. 

        \begin{center}
            \textbf{Personal Thoughts}
        \end{center}
        
        Like I said, right on the nose. There is nothing in this chapter that doesn't make sense to me. It's quite clear that how I used to be about other people, i.e. have that stupid hero complex, is nothing short of destructive. That doesn't mean that I can't help other people when they need it, but it does mean that I also have to take care of myself and put my needs first from time to time, or even most of the time. Sacrifice is exactly that... sacrifice, and if you keep giving parts of yourself, eventually you won't have anything left over to receive.
        
    \end{multicols}
    
\section{Make friends with people who want the best for you}
     \begin{multicols}{2}
        \begin{center}
            \textbf{Overview}
        \end{center}
        
        There are people in life who are going to get incredibly jealous and try to pull you back down to their level. There are also people who are going to try and pull you back because you're going down the wrong path. Determining the difference between those people is and incredibly tricky line, and it's one of the finest lines known to man. How do you tell the difference between care and jealousy when you're making progress in something? Progress is progress, whether the activity is productive or not... there is a reason people get really good at partying and drinking, and it's because they make consistent progress towards it. It's strange to think about.

        \begin{center}
            \textbf{Personal Thoughts}
        \end{center}
        
        Luckily for me, I don't seem to have trouble finding these types of people. They seem to be abundantly everywhere, and I think it's because I am one of those people that want the best for others and don't get jealous or angry (anymore) when they succeed. It's a liberating thing to see how when you do something, it can be reflected back at you for one reason or another.
        
    \end{multicols}

\section{Compare yourself to who you were yesterday, not to who someone else is today}
    \begin{multicols}{2}
        \begin{center}
            \textbf{Overview}
        \end{center}
        
        You are in charge of yourself. In the wise words of Bruce Lee, I am not here to live up to your expectations and you are not here to live up to mine. It's quite interesting how once you let yourself go and start focusing on your own progress and not the progress of others, you tend to start seeing linear and even exponential progress. 

        \begin{center}
            \textbf{Personal Thoughts}
        \end{center}
        
        This it true to many degrees. Watching TV, scrolling through Instagram... you don't realize how often you compare yourself to others until you start to weed yourself off of it, and you realize that everything that you've been doing has been filled with jealousy, hopelessness, and a sense of loss. It's quite interesting how in just a couple months of getting off of those systems, you go ahead and turn it all around in one go, and you can definitely keep it going for a while it seems like. I think that's an interesting concept, and it's not something that I've ever thought about until I was one of those drones. But now that I know, I can take measures to prevent it from ever happening again. And for the record... I'm loving it... Oh I'm in love with the progress. I didn't realize how much I missed the momentum, the wildly racing thoughts and ideas, and the ability to think this critically or make decisions on the fly. This is going to be something that I'll enjoy for a while, and I hope to keep loving it.
        
    \end{multicols}

\section{Do not let your children do anything that makes you dislike them}
        \begin{multicols}{2}
        \begin{center}
            \textbf{Overview}
        \end{center}
        
        It is you job to raise children into being people that you desire them to be, hopefully ones that are better than you and are able to sidestep the mistakes that you've made along the way. It's a tough world, and your children need to be able to handle it. You need to fill them with love, discipline, and the courage to think. It's your job to make them independent and able to achieve greatness on their own, rather than be dependent on you and subject to their every impulse and vice. It's a hard line as well when you need to punish... How do you decide the severity of the punishment? Is it physical? Is it just a stern look? The simple rule: The punishment should fit the crime.

        \begin{center}
            \textbf{Personal Thoughts}
        \end{center}
        
        I can't speak to this. I am not a father... as of writing this, I am 23 years old with no dating prospects in sight. I'm just hoping to be a good father one day, so that means right now I have to work to be a person that my children can look up to.
    \end{multicols}

\section{Set your house in perfect order before you criticize the world}
        \begin{multicols}{2}
        \begin{center}
            \textbf{Overview}
        \end{center}
        
        This has multiple points to it, but the main one is organization and discipline in the home. Before you can start to make the world a better place, you have to woork to set your own life in order, and that starts in the home.

        \begin{center}
            \textbf{Personal Thoughts}
        \end{center}
        
        I definitely struggle with this. The past two years I have done a poor job of keeping myself organized. I need to do better!
    \end{multicols}
    
\section{Pursue what is meaningful (not what is expedient)}
        \begin{multicols}{2}
        \begin{center}
            \textbf{Overview}
        \end{center}
        
        Your life is short, and you need to spend it finding what is meaningful to YOU, not what is quick and easy. Purpose and meaning are what defines life for any given person. It solidifies one's actions across your entire lifetime, and enables you to give everything that you are and everything that you've got into your passions. Your purpose in life is not to be happy, but to have meaning. When your life has meaning, you will not necessarily be happy, but you will be fulfilled beyone measure. Find your purpose, find your meaning, and you will have a story worth telling for eons to come.  

        \begin{center}
            \textbf{Personal Thoughts}
        \end{center}
        
        This hits deeper than anything that I ever could've imagined. Hearing the author speak these words was magnificent, because you could hear the cry for betterment through his actions in the thoughts that he has put on paper. It was wonderful to hear this, and showed how little I myself have thought about my meaning, my purpose. The more I think on it, the more that I believe I am destined to be someone that others look up to, that others go to when they are having a hard time, someone that is willing to be a guiding hand and role model in all parts of life.

        Due to this realization of my purpose, I see now that it is my civic duty to become great at anything that I do. Not only great, but one of the best, so that I can give the people around me permission to do the same. I HAVE to let my light shine through... else it would be a disservice to those around me.

        Woe is me I suppose. I have to put in the hours and put in the work all the time. But honestly... I forgot how much I love the grind, and how much I missed being in the situations that require me to push myself to my absolute limit and think outside the box. It's kind of mesmerizing to say the least, and hopefully everything that I do from now on I remember my purpose to the extreme.
    \end{multicols}
    
\section{Tell the truth – or, at least, don't lie}
        \begin{multicols}{2}
        \begin{center}
            \textbf{Overview}
        \end{center}
        
        There are mutliple truths to anything that you do, and that becomes apparent in highly emotional scenarios. The question "did you enjoy the show?" after watching a friends performance, you can say a couple of things, especially if you didn't like it. However, saying "I did not like the show" implies that you did not like their performance. Saying "I always enjoy watching you perform" is true, and should be spoken in that instance when emotions are running high. Then, it is your responsibility to go up to that person afterward and say "hey, I didn't like the show, here's why" when the emotions are not at a critical peak.

        \begin{center}
            \textbf{Personal Thoughts}
        \end{center}
        
        I actually believe this is something where I shine. I like to believe that I am pretty good at this... sidestepping my feelings in an effort to tell the truth, then coming back to it later.
    \end{multicols}
    
\section{Assume that the person you are listening to might know something you don't}
        \begin{multicols}{2}
        \begin{center}
            \textbf{Overview}
        \end{center}
        
        Many people nowadays believe they are the center of the universe and hold all knowledge available, which is a tragedy. When you are in conflict with each other, it is hard to sidestep past your own emotions because your truth ultimately is true. Chances are, you have your reasons for believing what it is you believe. You also have to remember in these scenarios that they also believe the same thing. 

        \begin{center}
            \textbf{Personal Thoughts}
        \end{center}
        
        I do tend to struggle here. I'm so thick skulled that it is crazy. I want the other person to understand what I'm saying, and in doing this I tend to negate what they are saying unintentionally. I don't mean to do it, but it happens, and I need to think a bit more about what I say before I say it.
    \end{multicols}

\section{Be precise in your speech}
        \begin{multicols}{2}
        \begin{center}
            \textbf{Overview}
        \end{center}
        
        Words can hurt. That about sums this up

        \begin{center}
            \textbf{Personal Thoughts}
        \end{center}
        
        Again. I need to think a little more about what I say before I say it. Nuf said
    \end{multicols}
    
\section{Do not bother children when they are skateboarding}
        \begin{multicols}{2}
        \begin{center}
            \textbf{Overview}
        \end{center}
        
        This is a strange way to introduce the concept of allowing people to be curious and take on danger in order to conquer their own demons, but its true. You should allow your kids to explore and find things out for themselves, even if it is potentially dangerous. Conquering that fear, creating kinship through that experience, is what leads boys to become men at the end of the day. It's what allows them to be adventurous and take on the world without fear of failure because they've learned to fail over and over again.

        \begin{center}
            \textbf{Personal Thoughts}
        \end{center}
        
        As a kid, my parents did a pretty good job of letting me explore and do things on my own. That certainly expressed itself in college, when I pretty much did what I want when I wanted, and no one could really tell me no if I set my mind to it. I've lost that in recent years as of writing this, but I'm slowly starting to get it back. After these past two years, the importance of voicing my opinion and standing my ground where I think it's necessary is more prevalant than ever before.
    \end{multicols}
    
\section{Pet a cat when you encounter one on the street}
        \begin{multicols}{2}
        \begin{center}
        \textbf{Overview}
        \end{center}
        
        The point of this chapter kind of flew over my head as I was listening to it, since I am young and brash, but further reflection on it makes it clear what he was saying: Appreciate the times that you have, in every moment, because who knows when it will be the last time. He talks about how the number of times that he will see his parents is numbered now that they are in their elderly years... that's a hard pill to swallow when you come to that realizations. Those visits, those small things over the course of life, should be appreciated with the fullness of heart that they deserve.

        \begin{center}
            \textbf{Personal Thoughts}
        \end{center}
        
        I've had the chance to think about this over my years in Detroit... how many times am I going to see those college friends over the rest of my lifetime? What about my family? My parents? What could happen in the next few months that skews that perception of that timeline into oblivion, and those numbers click down to 0 in an instant? It's scary stuff to think about. I've done myself the service of thinking about such things in the past few years before finding this book, but it's a stark reminder that what I've seen... all the death I've seen, all the LIFE that I've seen, comes and goes, and holding on to those moments and appreciating them at the time will make them so much more in the future when that's all they are in your brain: moments in time.
    \end{multicols}
\section{Personal Analysis}

This book has only emphasized the path I've set out on at the start of 2025... to become the best version of myself across all forms and beings that I can be: mentally, emotionally, spiritually, and physically. Without any of that, my purpose and meaning is for nought. I refuse to be a whisper in the dark anymore. I want to be the shout in the light that gives everyone else a reason to stand beside me and courageously pursue themselves no matter who they are.
\end{document}
