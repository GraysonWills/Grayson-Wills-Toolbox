\documentclass{article}
\usepackage{multicol}
\usepackage{ragged2e}
\usepackage{listings}
\title{Add Two Numbers}
\author{Grayson Wills}
\begin{document}
\section{My Solution}
\begin{lstlisting}
    class Solution:
    def twoSum(self, nums: List[int], target: int) -> List[int]:
        myMap = {}
        for index in range(0, len(nums)):
            if target - nums[index] not in myMap:
                myMap[nums[index]] = index 
                continue
            solution = [myMap[target - nums[index]], index]
            return solution
\end{lstlisting}
\section{Recommended Solution}
\begin{lstlisting}
    class Solution:
    def twoSum(self, nums: List[int], target: int) -> List[int]:
        numMap = {}
        n = len(nums)

        for i in range(n):
            complement = target - nums[i]
            if complement in numMap:
                return [numMap[complement], i]
            numMap[nums[i]] = i

        return []  
\end{lstlisting}
\section{Notes}
The solution that I provided take a couple extra lines of code, and does not appear to account for the empty case like the recommended solution does. We are allowed to assume that there is a solution and ONLY one solution, but this kind of programming allows for leeway when there are cases where this is no longer a valid assumption.

Additionally, the if statement asks for if the complement is currently in the map, whereas mine asks if it is NOT. This causes me a couple extra lines of code that I could remove. 

Initializing the solution also gave me an extra line that I can take out, although it is cleaner to see that way. I will continue to provide the solution in such a way, since adding the initializer doesn't cause any issues, it's not an extra step in the compiler, and for future reference it would be easier to change/understand in the future what is happening
\end{document}